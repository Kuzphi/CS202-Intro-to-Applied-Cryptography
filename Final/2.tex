\normalfont\documentclass[letterpaper,11pt]{article}
\usepackage{amsmath, amsfonts,amssymb,latexsym}
\usepackage{fullpage}
\usepackage{parskip}
\usepackage{flexisym}
\usepackage{algorithm}
\usepackage{indentfirst}
\usepackage{graphicx}
\usepackage{algorithmicx}
\usepackage{algpseudocode}
\usepackage{amsmath}
\begin{document}
\setlength{\parindent}{2ex}
\newcommand{\header}{
	\noindent \fbox{
	\begin{minipage}{6.4in}
  	\medskip
  	\textbf{CS 202 - Introduction to Applied Cryptography} \hfill \textbf{Fall 2016} \\[1mm]
  	\begin{center}
    	{\Large Final} \\[3mm]
  	\end{center}
	\today \hfill \itshape{Liangjian Chen}
	\medskip
	\end{minipage}}
}
\newcommand{\RN}[1]{%
  \textup{\uppercase\expandafter{\romannumeral#1}}%
}

\bigskip
\header

\begin{enumerate}
	\item[1.1]
	The prove broken where combine the euqation (11.3) and (11.6) to get (11.2). Because RSA plian scheme is not CPA-secure, so when adversary $\mathcal{A}$ observe the $Enc_{pk}(m_{1,0})$, it would change its behavior on second message pair. Thus equation (11.3) and (11.6) does not holds anymore. Therefore so is the rest proof.
\end{enumerate}
\end{document}
