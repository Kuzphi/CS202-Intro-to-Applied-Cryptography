\normalfont\documentclass[letterpaper,11pt]{article}
\usepackage{amsmath, amsfonts,amssymb,latexsym}
\usepackage{fullpage}
\usepackage{parskip}
\usepackage{flexisym}
\usepackage{algorithm}
\usepackage{indentfirst}
\usepackage{graphicx}
\usepackage{algorithmicx}
\usepackage{algpseudocode}
\usepackage{amsmath}
\begin{document}
\setlength{\parindent}{2ex}
\newcommand{\header}{
	\noindent \fbox{
	\begin{minipage}{6.4in}
  	\medskip
  	\textbf{CS 202 - Introduction to Applied Cryptography} \hfill \textbf{Fall 2016} \\[1mm]
  	\begin{center}
    	{\Large HomeWork 3} \\[3mm]
  	\end{center}
	\today \hfill \itshape{Liangjian Chen}
	\medskip
	\end{minipage}}
}
\newcommand{\RN}[1]{%
  \textup{\uppercase\expandafter{\romannumeral#1}}%
}

\bigskip
\header

\begin{enumerate}
\item [Problem 1]\textbf{Solution:}\par
	\begin{enumerate}
		\item
			Expansion rate: $|H(x)| = |G(1|x)| = 2|x| + 2$.\\
			Yes, it is a secure PRG.\\
			\textbf{Proof:}
			Suppose, there is a efficient attack $A$ against $H$. Then we use attack $A$ against $G$.
			Assume $p_A=|\text{Pr}[A(H(s)) = 1] - \text{Pr}[A(r) = 1]|$, $p_B = |\text{Pr}[B(G(s)) = 1] - \text{Pr}[B(r) = 1]|$. If the beginning of $s$ is $1$, $B$ is same with $A$. If th beginning of $s$ is $0$, $B$ is always incorrect. So $p_B = p_A / 2 + 0 / 2 = p_A/2$ which is still non-negligible. So it violets the assumption that $G$ is a secure PRG. Thus $H$ is a secure PRG.
		\item
			Expansion rate: $|H(x)| = |G(x_L|x_R)|G(x_R|x_L)| = 4|x|$.\\
			No, it is not a secure PRG.\\
			\textbf{Proof:}
			Assume it is a secure PRG, then construct $F = H(x_L|x_R)|H(x_R|x_L)$. According to assumption, $F$ is secure. However, $F = G(x_L|x_R)|G(x_R|x_L)|G(x_R|x_L)|G(x_L|x_R)$, the first and fourth quarter of bits are same, second and the third quarter of bits are same. So we can easy construct a $D$, which check the first, fourth quarter, and second and third quarter. Then $\text{Pr} = |1 - 2^{2n}|$ is non-negligible which contradict with $F$ is a secure PRG. So it is not a secure PRG.
		\item
			Expansion rate: $|H(x)| = |G(z_L)|G(z_R)| = 2|G(x)| / 2 * 2 = 4|x|$.\\
			Yes, it is a secure PRG.\\
			\textbf{Proof:}
			Suppose, there is a efficient attack $A$ against $H$. Construct attack $B$ against $G$ as follow:\\ Since knowing $G(x)$ now, we are able to calculate $G(G(x)_L)|G(G(x)_R)$. Then use $A$ to attack it. Since $H(x) = G(z_L)|G(z_R) = G(G(x)_L)|G(G(x)_R)$, $B$ is a efficient attack which contradict with the assumption. Thus $H$ is a secure PRG.
	\end{enumerate}
\item [Problem 2]\textbf{Solution:}\par
	Suppose we have an efficient algorithm $A$ to attack $H$. Then construct $B$ as follow:\\
	$B$ choose $m_0$ and $m_1$ and send them to $D$, and get the cipher-text $c$ from $D$. Then calculate $w_0 = m_0 \oplus c$. if $B(w_0)$ returns $1$, $A$ returns $0$, otherwise, $A$ returns $1$.
	If the $H$ in problem 1 is secure PRG, then its stream cipher is secure. Otherwise it is not secure.
\item [Problem 3]\textbf{Solution:}\par
	Assume attacker $A$ breaks MM-CPA of E. Construct sequence $D$ as follow, for every $i$, $m_0^i = m_1^{i-1}$, $b_i = i\%2$. According to hint, there must exist a $i$, that a distinguishes $D_i$ and $D_{i-1}$ with a probability $\epsilon^\prime_A = \epsilon_A/p(n)$. Since $D_i$ and $D_{i-1}$ differ on a single cipher-text($m^i_1$), the probability of the difference between $m^{i-1}_1$ and $m^i_0$ is non-negligible. Thus, in attack $A^\prime$, choose $m^\prime_0 = m^{i-1}_1$ and $m^\prime_1 = m^{i}_0$. Then $A^\prime$ is a efficient attacker of CPA of E. 
\end{enumerate}
\end{document}
