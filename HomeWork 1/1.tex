\normalfont\documentclass[letterpaper,11pt]{article}
\usepackage{amsmath, amsfonts,amssymb,latexsym}
\usepackage{fullpage}
\usepackage{parskip}
\usepackage{flexisym}
\usepackage{algorithm}
\usepackage{indentfirst}
\usepackage{graphicx}
\usepackage{algorithmicx}
\usepackage{algpseudocode}
\usepackage{amsmath}
\begin{document}
\setlength{\parindent}{2ex}
\newcommand{\header}{
	\noindent \fbox{
	\begin{minipage}{6.4in}
  	\medskip
  	\textbf{CS 202 - Introduction to Applied Cryptography} \hfill \textbf{Fall 2016} \\[1mm]
  	\begin{center}
    	{\Large HomeWork 1} \\[3mm]
  	\end{center}
	\today \hfill \itshape{Liangjian Chen}
	\medskip
	\end{minipage}}
}
\newcommand{\RN}[1]{%
  \textup{\uppercase\expandafter{\romannumeral#1}}%
}

\bigskip
\header

\begin{enumerate}
\item (Problem 1)\par
	\begin{enumerate}
	\item[1.1]
		As hint says, just consider about the case of $\ell = 1$
		\begin{enumerate}
		\item[a]
			Yes, through $\mathcal{K}$, all messages get the same ciphertext set $\mathcal{C} = \mathcal{S}^2$.\\
			$P(\mathcal{C}=c|\mathcal{M}=m) = \frac{1}{4}$, which is a constant. So, $\forall{m},P(\mathcal{C}=c|\mathcal{M}=m)$ are same.
		\item[b]
			No. because, $|\mathcal{K}| < |\mathcal{M}|$
		\item[c]
			No. $01 \oplus 10 = 11$, so $11 \in \mathcal{C}$. However there is not key $k$ that $Enc_k(00) = 11$. Violet the Shannon's theorem.
		\end{enumerate}
	\item[1.2]
		When $|\mathcal{K}| < |\mathcal{M}|$, which is in case(b), cipher could not be secure according to the \textbf{theorem 2.10} in the textbook. When $\mathcal{K} = \{0,1\}^\ell$, cipher is always secure, because no matter what message set is, as long as it is not empty, the ciphertext set $\mathcal{C}$ is always $\{0,1\}^\ell$ which makes $P(\mathcal{C}=c|\mathcal{M}=m) = 2^{-\ell}$.
	\end{enumerate}
\item (Problem 2)\par
	\begin{enumerate}
		\item  $\forall{m},P(\mathcal{C}=c|\mathcal{M}=m) = \frac{1}{26}$ is a constant, so they are all same.
		\item let take a example, $m_0=aa, m_1 = ab$ and $c = bb$. we can say $P(\mathcal{C}=c|\mathcal{M}=m_0) = \frac{1}{26}$ however $P(\mathcal{C}=c|\mathcal{M}=m_1) = 0$.
		\item this means this cipher is not perfectly secrecy anymore. The prior distribution and posterior distribution of $\mathcal{M}$ are different.
		\item the largest set of $A$ is $\mathcal{K}$.\\
		Proof:\\
		(1)First, I want to argue that if $\mathcal{M}=\mathcal{K} = A$, then $\forall{m \in \mathcal{M}}, A = \{Enc_k(m)|\forall k \in \mathcal{K}\}$,as well as the mapping $\forall{m \in \mathcal{M}}, \mathcal{F}: \mathcal{K} \to \{Enc_k(m)|\forall k \in \mathcal{K}\}$ is bijection.\\
		The proof is easy if aware that $A$ is a permutation group. Since $A$ is a subgroup of $A$, for any 	element $m$, its coset is $A$. Thus $A = \{Enc_k(m)|\forall k \in \mathcal{K}\}$ holds. Also because of coset, the mapping is bijection \\
		Now, two condition of Shannon's theorem are met. So it is a perfect cipher.\par
		(2)According to Shannon's theorem, $|\mathcal{M}| \leq |\mathcal{K}|$. The size of any possible set is less or equal to the size of $\mathcal{K}$.\par
		Base on (1),(2), A is a largest perfect cipher. 
	\end{enumerate}

\item (Problem 3)\par
	An example, $m_0=00, m_1 = 01$ and $c = 00$. Then $P(\mathcal{C}=c|\mathcal{M}=m_0) = \frac{1}{2}$, $P(\mathcal{C}=c|\mathcal{M}=m_1) = 0$.
\item (Problem 4)\par
	\begin{enumerate}
		\item[2.11]
			Let $\mathcal{M}$ be the set of all possible messages that are possible decode of c, that is,
			\begin{align*}
			\mathcal{M}(c) = \{m| m  = \text{Dec}_k(c) \text{ for someone }k \in \mathcal{K}\}
			\end{align*}
			Because for every different k, there are at most $2^t$ different choice about $m$, so $|\mathcal{M}(c)| \leq 2^t|\mathcal{K}|$. Thus $ |\mathcal{K}| \geq 2^{-t}|\mathcal{M}(c)|$
		\item[2.12]
			The idea is first filling some elements into set $\mathcal{K}$ to make its size equal to $\mathcal{M}$ and cipher is "perfect", then calculate the probability and subtract these elements out.\par
			Consider following encryption algorithm $\RN{3}$. Let $|\mathcal{K}| = |\mathcal{M}|$, and a special attack rule in adversarial experiment is that there is a special key set $K_{valid} \in \mathcal{K}$. When Gen generate a key which is \textbf{not} in this set, adversary win directly. Then, it leads following probability:
			\begin{align*} 
				Pr[\text{Privk}^{eav}_{\mathcal{A},\RN{3}} = 1] = \frac{M - K}{M} + \frac{1}{2} * \frac{K}{M}
			\end{align*}
			here, $M = |\mathcal{M}|, K = |K_{valid}|$. Now, we take other elements away and only keep the set $K_{valid}$ as $\mathcal{K}$. Still, we can see that new game is the same as previous one and the probability is not change. Which leads:
			\begin{align*} 
			Pr[\text{Privk}^{eav}_{\mathcal{A},\RN{3}} = 1]&\leq \frac{1}{2} + \epsilon\\
				\frac{K}{M} + \frac{1}{2} * \frac{M - K}{M} &\leq \frac{1}{2} + \epsilon\\
				K &\geq (1 - 2\epsilon)M
			\end{align*}
			Thus, the bound is $K \geq (1 - 2\epsilon)M$.
	\end{enumerate}
\end{enumerate}
\end{document}
